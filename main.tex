%!TEX program = xelatex
\documentclass[11pt, aspectratio=169]{beamer}
\usefonttheme{professionalfonts}

\usepackage{amsmath}
\usepackage{fontspec}
\usepackage{graphicx}
\usepackage{import}
\usepackage{kotex}
\PassOptionsToPackage{table}{xcolor}
\usepackage{calc}
\usepackage{listings}
\usepackage{indentfirst}
\usepackage{tabularx}
\usepackage{ulem}
\usepackage{multicol}
\usepackage{epigraph}
\usepackage[many]{tcolorbox}

\definecolor{boj}{RGB}{0,118,191}
\definecolor{ucpc-orange}{RGB}{255,153,0}
\definecolor{acgreen}{RGB}{0,159,107}
\definecolor{wared}{RGB}{231,76,60}

\definecolor{acbronze}{RGB}{173,86,0}
\definecolor{acsilver}{RGB}{67,95,122}
\definecolor{acgold}{RGB}{236,154,0}
\definecolor{acplatinum}{RGB}{39,226,164}
\definecolor{acdiamond}{RGB}{0,180,252}
\definecolor{acruby}{RGB}{255,0,98}

\setbeamercolor{title}{fg=black}
\setbeamercolor{frametitle}{fg=ucpc-orange}
\setbeamercolor{structure}{fg=ucpc-orange}

\linespread{1.2}
\everymath{\displaystyle}

\graphicspath{ {./images/} }
\lstset{basicstyle=\footnotesize\ttfamily,breaklines=true}

\newcommand{\translation}[1]{\textsuperscript{#1}}

\setlength\fboxsep{0pt}

\newcommand{\complexity}[1]{$\mathcal{O}\left({#1}\right)$}
\newcommand{\difficulty}[1]{\includegraphics[width=1em,natwidth=1000,natheight=1000]{#1.svg.png}}
\newcommand{\norm}[1]{\left\lVert#1\right\rVert}

\usetheme{Ucpc2020}
\usetikzlibrary{arrows.meta,matrix,decorations.pathreplacing}

\title{2024 NPC 모의대회\#0 - Div1+2}
\subtitle{Official Solutions}
\author{소프트웨어 19 박병규}
\date{2024년 5월 2일}

\begin{document}
    \setcounter{framenumber}{-1}
    \frame{\titlepage}

    \begin{frame}{\textbf{공지사항}}
        \begin{itemize}
            \item solved.ac 티어, 알고리즘 분류를 꺼주세요. (설정->보기->보지 않기로 변경)
            \item 문제 지문은 모두 영어이며, 문제 번호는 난이도 순입니다.
            \item 대회 중 모든 검색은 허용됩니다.
            \item 난이도 커브는 Codeforces Div2와 Div3 사이입니다.
        \end{itemize}
    \end{frame}
        
    \begin{frame} % No title at first slide
        \begin{center}
            \begin{tabular}{cl|l|l}
                \hline
                문제 & & 난이도 & 알고리즘 \\
                \hline
                \hline
                \textbf{A} & Streets Ahead & \textbf{\color{acsilver}Silver5} & \texttt{set/map} \\
                \textbf{B} & Code Guessing & \textbf{\color{acsilver}Silver5} & \texttt{bruteforce, case\_work} \\
                \textbf{C} & Corrupted Gradebook & \textbf{\color{acgold}Gold4-5} & \texttt{dynamic\_programming} \\
                \textbf{D} & Shuffles & \textbf{\color{acgold}Gold4} & \texttt{math, adhoc} \\
                \textbf{E} & ABC String & \textbf{\color{acgold}Gold1} & \texttt{greedy} \\
                \textbf{F} & Hopscotch & \textbf{\color{acplatinum}Platinum3} & \texttt{math, combinatorics, dnc} \\
                \hline
            \end{tabular}
        \end{center}
    \end{frame}
    \import{solutions/}{A-Streets-Ahead.tex}
    \import{solutions/}{B-Code-Guessing.tex}
    \import{solutions/}{C-Corrupted-Gradebook.tex}
    \import{solutions/}{D-Shuffles.tex}
    \import{solutions/}{E-ABC-String.tex}
    \import{solutions/}{F-Hopscotch.tex}
\end{document}
